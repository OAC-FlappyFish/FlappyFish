\documentclass[notitlepage]{report}
\usepackage[portuguese]{babel}
\usepackage[utf8]{inputenc}
\usepackage{graphicx}
\usepackage{url}
\usepackage{hyperref}
\providecommand{\keywords}[1]{\textbf{\textit{Palavras chave:}} #1}

% Title Page
\title{Universidade de Brasília \\ Organização e Arquitetura de Computadores \ \\ Turma A - Prof. Marcus Vinicius Lamar \\ \textbf{Projeto Aplicativo} - Relatório Técnico   - Grupo \textbf{5} }
\author{João Vitor A.Ribeiro - 12/0014491\\Yurick Hauschild Caetano da Costa - 12/0024136\\João Henrique Sousa - 11/0014171\\Pedro Paulo Struck Lima - 11/0073983}


\begin{document}
\maketitle

\begin{abstract}
{Relatório do projeto final da disciplina \textbf{Organização e Arquitetura e Computadores} da Universidade de Brasília realizado no 1º semestre de 2014. Desenvolver um jogo estilo \textit {Flappy Bird} utilizando o processador MIPS Pipeline v4 fornecido e incrementado ao longo do curso para implementação em FPGA Altera DE2-70.} \\

\keywords{flappy,bird,mips,assembly,fpga}

\end{abstract}
\pagebreak{}


\chapter{Introdução}
\section{Arquitetura MIPS \& Hardware}
\paragraph{}{MIPS  - \textit{Microprocessor without interlocked pipeline stages}  é uma simples, reduzida, e altamente escalonável arquitetura \textbf{RISC}. Uma arquitetura \textbf{RISC} é uma arquitetura que possui um conjunto reduzido de instruções. Ao longo dos anos a arquitetura MIPS evoluiu e incorporou novas tecnologias, desenvolvendo um ambiente robusto e um apoio industrial considerável. Algumas de suas características principais são: alto número de registradores, número reduzido de instruções e o aspecto do seu conjunto de instruções, juntamente com a visibilidade do atraso dos estágios de \textit{pipeline}.}\cite {IT14}

\paragraph{}{Uma implementação de um microprocessador MIPS de $ 32 $ bits será utilizada como base de \textit{Hardware}. Esta base será constituída da implementação na linguagem \textbf{Verilog}, uma linguagem de descrição de hardware,  em uma placa \textbf{FPGA} - \textit{Field-programmable gate array }, um circuito integrado projetado para ser configurado e programado por um usuário.}

\section{Flappy Bird}
{Flappy Bird é um jogo criado por Dong Nguyen em 2013 por meio do estúdio \textit{.GEARS.}, um estúdio de desenvolvimento de jogos independente localizado no Vietnam. \cite{FBIRD14} É proposta a codificação e implantação de um jogo com objetivos, lógica e interface gráfica semelhantes. É possível visualizar uma amostra do jogo no endereço eletrônico \url{(http://flappybird.io/)}.



\chapter{Objetivos}
\begin {itemize}
\item Aplicação de conhecimentos e habilidades adquiridas durante o curso da disciplina \textbf{Organização e Arquitetura e Computadores} no desenvolvimento de um jogo similar ao \textit {Flappy Bird}.
\item Utilização de conhecimentos de programação em linguagem \textit{Assembly MIPS} - \textit{Software (Arquitetura)}
\item Utilização de conceitos de projetos básicos de microprocessadores -\textit{Hardware (Organização)}
\end{itemize}

\chapter{Metodologia}
\section{Metodologia utilizada}
\paragraph{}{}
{Foi adotada uma metodologia de desenvolvimento iterativo e incremental. Reuniões foram realizadas entre os membros do grupo para definição do planejamento, projeto e acompanhamento da implantação.}
\section{Divisão de tarefas}
\paragraph{}{}
{Foi adotada para a execução do projeto uma metodologia de segregação de funções entre os membros do grupo. Esta divisão se deu pela separação nas seguintes áreas:}
\begin{enumerate}
\item Área de organização do microprocessador (MIPS Pipeline) utilizado e \textit{hardware};
\item Área de codificação, mídia e \textit{software};
\item Documentação;
\item Interfaceamento, testes e garantia de qualidade.
\end{enumerate}




\section{Ferramentas utilizadas}

\subsection{Kit de desenvolvimento DE2-70}
\paragraph{}{Utilização do kit de desenvolvimento DE2-70 da Altera® que inclui:
\begin{itemize}
\item DE2-70
\item Cabo USB para programação e controle da FPGA
\item Fonte de alimentação 12V DC
\end{itemize} }

\subsection{Simulador MARS}
\paragraph{}{Utilização da ferramenta MARS - \textit{MIPS Assembler and Runtime Simulator} desenvolvida pela \textbf{Missoury State University}.\cite{MARS13}. Será utilizada a versão 4.4, com extensões adicionadas por contribuidores ao redor do mundo.}

\subsection{Software Quartus II Web Edition}
\paragraph{}{Utilização do software de projeto e design em FPGA's desenvolvido pela Altera® na versão $ 13.1 $ para  para implementação e implantação do processador MIPS. \cite {QUARTUS14}}

\subsection{Sistema de controle de versão Git}
\paragraph{}{Utilização da ferramenta Git para efetuar controle de versão na área de codificação. \cite {GIT14}}

\chapter{Problemas encontrados}
\paragraph{}{Problemas encontrados foram divididos conforme especificado na divisão proposta na seção (3.2.):}

\section{Organização do microprocessador e \textit{hardware}}
\section{Codificação, mídia e \textit{software}}
\section{Documentação}
\section{Interfaceamento, testes e garantia de qualidade}


\chapter{Resultados obtidos}
\paragraph{}{Pode-se dizer que isso isso e isso, e os resultados foram isso isso e isso. \\

--LINK VÍDEO--}

\chapter {Considerações finais}
\section{Conclusões}
\section{Trabalhos futuros}
\paragraph{}{Não há planos para trabalhos futuros envolvendo as problemáticas abordadas nesse trabalho.}

\begin{thebibliography}{1}
\bibitem[IT14]{IT14} Imagination Technologies.
\textit {MIPS Architectures} 2014. Endereço eletrônico. Acesso em 22 jun, 2014 Disponível em \url {http://www.imgtec.com/mips/architectures/}

\bibitem[FBIRD14]{FBIRD14}(Vietnamita) Thanh Nien. Chàng trai viet game Flappy Bird gây sõt toàn cãu". Acesso em 22 jun, 2014. Disponível em \url{http://www.thanhnien.com.vn/pages/20140206/chang-trai-viet-game-flappy-bird-gay-sot-toan-cau.aspx}

\bibitem[MARS13]{MARS13} Missouri State University. \textit{MARS (MIPS Assembler and Runtime Simulator): An IDE for MIPS Assembly Language Programming}. Endereço eletrônico. Acesso em 22 jun, 2014. Disponível em 
\url{http://courses.missouristate.edu/kenvollmar/mars/}

\bibitem[QUARTUS14]{QUARTUS14} Altera Corporation. \textit{Quartus II Web Edition}. Endereço eletrônico. Acesso em 22 jun, 2014. Disponível em 
\url{http://www.altera.com/products/software/quartus-ii/web-edition/qts-we-index.html}

\bibitem[GIT14]{GIT14} \textit{Git}. Endereço eletrônico. Acesso em 22 jun, 2014. Disponível em \url{http://git-scm.com/}
\end{thebibliography}



\end{document}   