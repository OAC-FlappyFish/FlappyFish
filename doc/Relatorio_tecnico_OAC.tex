\documentclass[notitlepage]{report}
\usepackage[portuguese]{babel}
\usepackage[utf8]{inputenc}
\usepackage{graphicx}
\usepackage{url}
\usepackage{hyperref}
\providecommand{\keywords}[1]{\textbf{\textit{Palavras chave:}} #1}

% Title Page
\title{Universidade de Brasília \\ Organização e Arquitetura de Computadores \ \\ Turma A - Prof. Marcus Vinicius Lamar \\ \textbf{Projeto Aplicativo} - Relatório Técnico   - Grupo \textbf{5} }
\author{João Vitor A.Ribeiro - 12/0014491\\João Henrique Sousa - 11/0014171\\Pedro Paulo Struck Lima - 11/0073983\\Yurick Hauschild Caetano da Costa - 12/0024136}


\begin{document}
\maketitle

\begin{abstract}
{Relatório do projeto final da disciplina \textbf{Organização e Arquitetura e Computadores} da Universidade de Brasília realizado no 1º semestre de 2014. Desenvolver um jogo estilo \textit {Flappy Bird} utilizando o processador MIPS Uniciclo v6 fornecido e incrementado ao longo do curso para implementação em FPGA Altera DE2-70.} \\

\keywords{flappy,bird,mips,assembly,fpga}

\end{abstract}
\pagebreak{}


\chapter{Introdução}
{}


\chapter{Objetivos}
\begin {itemize}
\item Aplicação de conhecimentos e habilidades adquiridas durante o curso da disciplina \textbf{Organização e Arquitetura e Computadores} no desenvolvimento de um jogo similar ao \textit {Flappy Bird} {(http://flappybird.io/)}.
\end{itemize}

\chapter{Metodologia}
{metodo de desenvolvimento,ferramentas utilizadas,divisão de trabalho}

\chapter{Problemas encontrados}
{asdasas? adsasda.}

\chapter{Resultados obtidos}
{sdaslkdasdçkasd}

\chapter{Considerações finais}
{foi muito legal trabalhar no projeto}

\chapter{Trabalhos futuros}
{Utilização dos conceitos utilizados na vida pessoal,academica}

\begin{thebibliography}{1}
\bibitem NJGNFJHNFKGNK
\bibitem RETERTETEW
\end{thebibliography}



\end{document}   